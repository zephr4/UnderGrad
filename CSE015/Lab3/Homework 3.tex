\documentclass[11pt]{article}

\usepackage[margin= 1in]{geometry}
\usepackage{amsmath}
\title{Homework 2}
\author{Andy Alvarenga}
\date{\today}

\begin{document}
\maketitle

\begin{enumerate}
\item
P is a Knight and Q is a Knave.\\
There are four potential cases:\\
\begin{enumerate}
\item $P = $knight and $Q = $knight
\item $P = $knave and $Q = $knight
\item $P = $knight and $Q = $knave
\item $P = $knave and $Q = $knave
\end{enumerate}
Case 1 is impossible beacuse if P is a knight then Q must be a knave.\\
In cases 2 and 4, if P is a knave, then what he says is untrue, therefore they are impossible.\\
This leaves only case 3, which is the only possible combination because if P is a knight then Q must be a knave since knights do not lie.\\

\item
A is Knave and B is a Knight.\\
There are four potential cases:\\
\begin{enumerate}
\item $A = $knight and $B = $knight
\item $A = $knave and $B = $knight
\item $A = $knight and $B = $knave
\item $A = $knace and $B = $knave
\end{enumerate}
Case 1 and 3 are impossible because if A is a knight, then whatever he says is true.\\
Case 4 is impossible because if B is a knave, he always lies.\\
Therefore, case 2 is the only possible combination. If A is knave then B must be a knight.\\
\end{enumerate}

\section{Logical Identities} 

\begin{enumerate}
\item 
$\neg(p \rightarrow (q \rightarrow p))$\\
~=~$\neg p \rightarrow  \neg(q \rightarrow p)$\\
~=~$\neg p \rightarrow (\neg q \rightarrow \neg p)$\\

\item
$\neg((p \land q) \rightarrow (q \lor p))$\\
~=~$\neg(p \land q) \rightarrow \neg(q \lor p)$\\
~=~$(\neg p \lor \neg q) \rightarrow (\neg q \land \neg p)$\\
\end{enumerate}

\section{Logical Equivilances}

\begin{enumerate}
\item
\begin{tabular}{|c|c|c|c|c|c|}
\hline
$p$ & $q$ & $r$ & $p \to (q \to r)$ & $(p \land q) \to r$ & $(p \to (q \to r)) \leftrightarrow  ((p \land q) \to r)$ \\
\hline
0 & 0 & 0 & 1 & 1 & 1 \\
0 & 0 & 1 & 1 & 1 & 1 \\
0 & 1 & 0 & 1 & 1 & 1 \\
0 & 1 & 1 & 1 & 1 & 1 \\
1 & 0 & 0 & 1 & 1 & 1 \\
1 & 0 & 1 & 1 & 1 & 1 \\
1 & 1 & 0 & 0 & 0 & 1 \\
1 & 1 & 1 & 1 & 1 & 1 \\
\hline
\end{tabular}
\newline
The propositional statements are equivilant beacause the final column of the truth table recieves a value of one for all cases, meaning $(p \to (q \to r)) \leftrightarrow  ((p \land q) \to r)$ is always true. Therefore, the pair is equivilant.

\item
\begin{tabular}{|c|c|c|c|c|c|}
\hline
$p$ & $q$ & $r$ & $p \to (q \to r)$ & $(p \to q) \to r$ & $(p \to (q \to r)) \leftrightarrow ((p \to q) \to r)$ \\
\hline
0 & 0 & 0 & 1 & 0 & 0 \\
0 & 0 & 1 & 1 & 1 & 1 \\
0 & 1 & 0 & 1 & 0 & 0 \\
0 & 1 & 1 & 1 & 1 & 1 \\
1 & 0 & 0 & 1 & 1 & 1 \\
1 & 0 & 1 & 1 & 1 & 1 \\
1 & 1 & 0 & 0 & 0 & 1 \\
1 & 1 & 1 & 1 & 1 & 1 \\
\hline
\end{tabular}
\newline
The propositional statements are not equivilant beacuse the final column of the truth table, $(p \to (q \to r)) \leftrightarrow ((p \to q) \to r)$, does not recieve a value of one for all cases.
\end{enumerate}

\section{Logical Consequence}

\begin{enumerate}
\item
This is a valid argument because we have no way of knowing if the conclusion is false or not.
\item
This is a valid argument since Puerto Rico is surrounded by water and since all islands are surrounded by water, then Puerto Rico must be an island.
\end{enumerate}

\section{Collaboration}
Alberc Ej Salcedo and Spencer Tang
\end{document}